In this paper, we presented a broadcast data stream allocation technique
(DFDI) that utilize the R-Tree and performs a depth-first traversal of the
index to create a distributed index. The goal of DFDI are to facilitate
query processing from broadcast data, to reduce index overhead (IP) and,
to improve the initial index prob. DFDI is able to achieve all these with
reasonable efficiency. DFDI is a flexible R-Tree based index and
well-supports skyline queries as well as other data query types. The allocation
distributes $b^h$ number of index segments among the broadcast program to
reduce initial index at the same time keeps the index overhead low. The simulation
results for index percentage show DFDI performs very well with 2 levels of
replication and follow the efficiency of one-time index with only $16\%$
increase in index overhead.

The simulation result show that the approach also performs well with data
of higher dimensions. The index overhead decreases as the number of records
remain constant and the number of data dimension increase. This is due to
the growth of dimensionality does not make the index tree grow "taller" and
does not incur the cost of new nodes when the index grows. The height of
an index tree does increase as the number of records increase, but as seen
in Figure~\ref{fig:ip_rc}, the growth of the index is not as fast as the
growth of the amount of data; therefore, the index overhead decreases as
records increase.

In addition, we introduced point-based and index-based pruning skyline
algorithms. The experiments shows both algorithms are capable of evaluate
skyline queries of combined $min$ and $max$ attributes with reasonable
tuning time and dominance tests. The index-based skyline had
always performed better than the point-based skyline, in some cases several
factors. The performance of the algorithms is also affected by the data
arrangement and R-Tree implementation. From our simulation, we find that
R-Tree that index records with lower attributes first, performs better for
$min$ skyline queries, and vice versa.